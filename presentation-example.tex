\documentclass{beamer}

% To avoid problems with .out files
\usepackage{bookmark}

% Overall theme
\usetheme{Madrid}

% Enable default preamble
% Setup
\usepackage[utf8]{inputenc}

% Math
\usepackage{amsmath, amssymb, mathtools}
\usepackage{xspace} % for making spacing nicer in macros

\newcommand{\xor}{\oplus}
\newcommand{\beq}{\stackrel{?}{=}}
\renewcommand{\mod}{\text{ mod }}
\newcommand{\QED}{\ensuremath{\square}}

\newcommand{\mc}[1]{\ensuremath{\mathcal{#1}}\xspace}
\newcommand{\mb}[1]{\ensuremath{\mathbb{#1}}\xspace}

%% Mathcal Chars
\newcommand{\Ac}{\mc{A}}
\newcommand{\Bc}{\mc{B}}
\newcommand{\Cc}{\mc{C}}
\newcommand{\Dc}{\mc{D}}
\newcommand{\Ec}{\mc{E}}
\newcommand{\Fc}{\mc{F}}
\newcommand{\Gc}{\mc{G}}
\newcommand{\Hc}{\mc{H}}
\newcommand{\Ic}{\mc{I}}
\newcommand{\Jc}{\mc{J}}
\newcommand{\Kc}{\mc{K}}
\newcommand{\Lc}{\mc{L}}
\newcommand{\Mc}{\mc{M}}
\newcommand{\Nc}{\mc{N}}
\newcommand{\Oc}{\mc{O}}
\newcommand{\Pc}{\mc{P}}
\newcommand{\Qc}{\mc{Q}}
\newcommand{\Rc}{\mc{R}}
\newcommand{\Sc}{\mc{S}}
\newcommand{\Tc}{\mc{T}}
\newcommand{\Uc}{\mc{U}}
\newcommand{\Vc}{\mc{V}}
\newcommand{\Wc}{\mc{W}}
\newcommand{\Xc}{\mc{X}}
\newcommand{\Yc}{\mc{Y}}
\newcommand{\Zc}{\mc{Z}}

%% Mathbb Chars
\newcommand{\Ab}{\mb{A}}
\newcommand{\Bb}{\mb{B}}
\newcommand{\Cb}{\mb{C}}
\newcommand{\Db}{\mb{D}}
\newcommand{\Eb}{\mb{E}}
\newcommand{\Fb}{\mb{F}}
\newcommand{\Gb}{\mb{G}}
\newcommand{\Hb}{\mb{H}}
\newcommand{\Ib}{\mb{I}}
\newcommand{\Jb}{\mb{J}}
\newcommand{\Kb}{\mb{K}}
\newcommand{\Lb}{\mb{L}}
\newcommand{\Mb}{\mb{M}}
\newcommand{\Nb}{\mb{N}}
\newcommand{\Ob}{\mb{O}}
\newcommand{\Pb}{\mb{P}}
\newcommand{\Qb}{\mb{Q}}
\newcommand{\Rb}{\mb{R}}
\newcommand{\Sb}{\mb{S}}
\newcommand{\Tb}{\mb{T}}
\newcommand{\Ub}{\mb{U}}
\newcommand{\Vb}{\mb{V}}
\newcommand{\Wb}{\mb{W}}
\newcommand{\Xb}{\mb{X}}
\newcommand{\Yb}{\mb{Y}}
\newcommand{\Zb}{\mb{Z}}

% Plots
\usepackage{pgfplots}
\pgfplotsset{compat = newest}
\NewDocumentCommand{\plot}{O{-2} O{10} O{-2} O{10} m}{
  \begin{center}
    \begin{tikzpicture}
      \begin{axis}[
        xmin=#1, xmax=#2, ymin=#3, ymax=#4, domain=#1:#2,
        xticklabel=\empty, yticklabel=\empty,
        minor tick num=1, axis lines = middle,
        xlabel=$x$, ylabel=$y$,
        label style = {at={(ticklabel cs:1.1)}}]
          \addplot[
              samples = 300,
          ] {#5};
      \end{axis}
    \end{tikzpicture}
  \end{center}
}

% Color
\usepackage{xcolor}

% Images
\usepackage{graphicx}
\newcommand{\img}[2][0.5]{
    \begin{center}
        \includegraphics[scale=#1]{#2}
    \end{center}
}

% Common text commands
\newcommand{\I}[1]{\textit{#1}}
\newcommand{\B}[1]{\textbf{#1}}
\newcommand{\U}[1]{\underline{#1}}
\newcommand{\T}[1]{\texttt{#1}}
\newcommand{\X}[1]{\sout{#1}}

% Special sections
\newcommand{\todo}{\textbf{\color{red} TODO}}
\newcommand{\comment}[1]{\textbf{\color{gray} Comment: #1}}

% Lists
\usepackage{enumitem} % for custom list enumeration

% TikZ
\usetikzlibrary{arrows}
\usetikzlibrary{shapes}
\usetikzlibrary{automata}

% Tables 
\usepackage{float}

% Bibliography
\usepackage[square,numbers]{natbib}
\bibliographystyle{abbrvnat}

% Links 
\usepackage{hyperref}
\hypersetup{
    colorlinks=true,
    linkcolor=black,
    filecolor=magenta,      
    urlcolor=blue,
}

% Email
\newcommand{\email}[2]{
  \href{mailto:#1}{#2}
}

% Make title page simple
\setbeamertemplate{title page}[default]
\setbeamercolor{titlelike}{parent=structure,bg=white}

% Make blocks rectangular
\setbeamertemplate{blocks}[default]

% Make listings look nice
\setlist[itemize]{label= \color{darkgray} $\bullet$}
\setlist[enumerate]{label= \color{darkgray} \arabic*.}

% Fix hyperref interference with beamer footline
\setbeamertemplate{footline}{%
  \leavevmode
  \hbox{%
    \begin{beamercolorbox}[
      wd=.333333\paperwidth,
      ht=2.25ex,
      dp=1ex,
      center
    ]{author in head/foot}%
      \usebeamerfont{author in head/foot}%
      \insertshortauthor
    \end{beamercolorbox}%
    \begin{beamercolorbox}[
      wd=.333333\paperwidth,
      ht=2.25ex,
      dp=1ex,
      center
    ]{title in head/foot}%
      \usebeamerfont{title in head/foot}%
      \hypersetup{hidelinks}%
      \insertshorttitle
    \end{beamercolorbox}%
    \begin{beamercolorbox}[
      wd=.333333\paperwidth,
      ht=2.25ex,
      dp=1ex,
      center
    ]{date in head/foot}%
      \usebeamerfont{date in head/foot}%
      \insertshortdate
      \hspace*{10ex}
      \insertframenumber{} / \inserttotalframenumber\hspace*{1ex}
    \end{beamercolorbox}%
  }
  \par
  \vspace{0pt}%
}
% Code 
\usepackage{listings}

\definecolor{codegreen}{rgb}{0,0.6,0}
\definecolor{codegray}{rgb}{0.5,0.5,0.5}
\definecolor{codepurple}{rgb}{0.58,0,0.82}
\definecolor{backcolour}{rgb}{0.95,0.95,0.92}

\lstdefinestyle{mystyle}{
    backgroundcolor=\color{backcolour},   
    commentstyle=\color{codegreen},
    keywordstyle=\color{magenta},
    numberstyle=\tiny\color{codegray},
    stringstyle=\color{codepurple},
    basicstyle=\ttfamily\footnotesize,
    breakatwhitespace=false,         
    breaklines=true,                 
    captionpos=b,                    
    keepspaces=true,                 
    numbers=left,                    
    numbersep=5pt,                  
    showspaces=false,                
    showstringspaces=false,
    showtabs=false,                  
    tabsize=2
}

\lstset{
    basicstyle=\ttfamily,
    mathescape,
    style=mystyle
}

% Add keywords to languages
\newcommand{\addkeywords}[2]{
    \lstdefinelanguage{My#1}[]{#1}{
        morekeywords={#2}
    }
}

\addkeywords{TeX}{
    title, 
    subtitle, 
    author, 
    date, 
    begin, 
    maketitle, 
    today, 
    end, 
    documentclass
}


\title{Preambles in \LaTeX}
\subtitle{For Presentations}
\author{Anders B. Clausen}
\date{\today}

\begin{document}
\frame{
  \titlepage
  \begin{center}
    \tiny \url{https://github.com/anbclausen}
  \end{center}
}

\frame{
  \tableofcontents
}

\section{Setup}
\begin{frame}[fragile]{Setup}
  Here is how to get going with a presentation:
  \begin{lstlisting}[language=MyTeX]
\documentclass{beamer}

% To avoid problems with .out files
\usepackage{bookmark}

% Overall theme
\usetheme{Madrid}

% Enable default preamble
% Setup
\usepackage[utf8]{inputenc}

% Math
\usepackage{amsmath, amssymb, mathtools}
\usepackage{xspace} % for making spacing nicer in macros

\newcommand{\xor}{\oplus}
\newcommand{\beq}{\stackrel{?}{=}}
\renewcommand{\mod}{\text{ mod }}
\newcommand{\QED}{\ensuremath{\square}}

\newcommand{\mc}[1]{\ensuremath{\mathcal{#1}}\xspace}
\newcommand{\mb}[1]{\ensuremath{\mathbb{#1}}\xspace}

%% Mathcal Chars
\newcommand{\Ac}{\mc{A}}
\newcommand{\Bc}{\mc{B}}
\newcommand{\Cc}{\mc{C}}
\newcommand{\Dc}{\mc{D}}
\newcommand{\Ec}{\mc{E}}
\newcommand{\Fc}{\mc{F}}
\newcommand{\Gc}{\mc{G}}
\newcommand{\Hc}{\mc{H}}
\newcommand{\Ic}{\mc{I}}
\newcommand{\Jc}{\mc{J}}
\newcommand{\Kc}{\mc{K}}
\newcommand{\Lc}{\mc{L}}
\newcommand{\Mc}{\mc{M}}
\newcommand{\Nc}{\mc{N}}
\newcommand{\Oc}{\mc{O}}
\newcommand{\Pc}{\mc{P}}
\newcommand{\Qc}{\mc{Q}}
\newcommand{\Rc}{\mc{R}}
\newcommand{\Sc}{\mc{S}}
\newcommand{\Tc}{\mc{T}}
\newcommand{\Uc}{\mc{U}}
\newcommand{\Vc}{\mc{V}}
\newcommand{\Wc}{\mc{W}}
\newcommand{\Xc}{\mc{X}}
\newcommand{\Yc}{\mc{Y}}
\newcommand{\Zc}{\mc{Z}}

%% Mathbb Chars
\newcommand{\Ab}{\mb{A}}
\newcommand{\Bb}{\mb{B}}
\newcommand{\Cb}{\mb{C}}
\newcommand{\Db}{\mb{D}}
\newcommand{\Eb}{\mb{E}}
\newcommand{\Fb}{\mb{F}}
\newcommand{\Gb}{\mb{G}}
\newcommand{\Hb}{\mb{H}}
\newcommand{\Ib}{\mb{I}}
\newcommand{\Jb}{\mb{J}}
\newcommand{\Kb}{\mb{K}}
\newcommand{\Lb}{\mb{L}}
\newcommand{\Mb}{\mb{M}}
\newcommand{\Nb}{\mb{N}}
\newcommand{\Ob}{\mb{O}}
\newcommand{\Pb}{\mb{P}}
\newcommand{\Qb}{\mb{Q}}
\newcommand{\Rb}{\mb{R}}
\newcommand{\Sb}{\mb{S}}
\newcommand{\Tb}{\mb{T}}
\newcommand{\Ub}{\mb{U}}
\newcommand{\Vb}{\mb{V}}
\newcommand{\Wb}{\mb{W}}
\newcommand{\Xb}{\mb{X}}
\newcommand{\Yb}{\mb{Y}}
\newcommand{\Zb}{\mb{Z}}

% Plots
\usepackage{pgfplots}
\pgfplotsset{compat = newest}
\NewDocumentCommand{\plot}{O{-2} O{10} O{-2} O{10} m}{
  \begin{center}
    \begin{tikzpicture}
      \begin{axis}[
        xmin=#1, xmax=#2, ymin=#3, ymax=#4, domain=#1:#2,
        xticklabel=\empty, yticklabel=\empty,
        minor tick num=1, axis lines = middle,
        xlabel=$x$, ylabel=$y$,
        label style = {at={(ticklabel cs:1.1)}}]
          \addplot[
              samples = 300,
          ] {#5};
      \end{axis}
    \end{tikzpicture}
  \end{center}
}

% Color
\usepackage{xcolor}

% Images
\usepackage{graphicx}
\newcommand{\img}[2][0.5]{
    \begin{center}
        \includegraphics[scale=#1]{#2}
    \end{center}
}

% Common text commands
\newcommand{\I}[1]{\textit{#1}}
\newcommand{\B}[1]{\textbf{#1}}
\newcommand{\U}[1]{\underline{#1}}
\newcommand{\T}[1]{\texttt{#1}}
\newcommand{\X}[1]{\sout{#1}}

% Special sections
\newcommand{\todo}{\textbf{\color{red} TODO}}
\newcommand{\comment}[1]{\textbf{\color{gray} Comment: #1}}

% Lists
\usepackage{enumitem} % for custom list enumeration

% TikZ
\usetikzlibrary{arrows}
\usetikzlibrary{shapes}
\usetikzlibrary{automata}

% Tables 
\usepackage{float}

% Bibliography
\usepackage[square,numbers]{natbib}
\bibliographystyle{abbrvnat}

% Links 
\usepackage{hyperref}
\hypersetup{
    colorlinks=true,
    linkcolor=black,
    filecolor=magenta,      
    urlcolor=blue,
}

% Email
\newcommand{\email}[2]{
  \href{mailto:#1}{#2}
}

% Make title page simple
\setbeamertemplate{title page}[default]
\setbeamercolor{titlelike}{parent=structure,bg=white}

% Make blocks rectangular
\setbeamertemplate{blocks}[default]

% Make listings look nice
\setlist[itemize]{label= \color{darkgray} $\bullet$}
\setlist[enumerate]{label= \color{darkgray} \arabic*.}

% Fix hyperref interference with beamer footline
\setbeamertemplate{footline}{%
  \leavevmode
  \hbox{%
    \begin{beamercolorbox}[
      wd=.333333\paperwidth,
      ht=2.25ex,
      dp=1ex,
      center
    ]{author in head/foot}%
      \usebeamerfont{author in head/foot}%
      \insertshortauthor
    \end{beamercolorbox}%
    \begin{beamercolorbox}[
      wd=.333333\paperwidth,
      ht=2.25ex,
      dp=1ex,
      center
    ]{title in head/foot}%
      \usebeamerfont{title in head/foot}%
      \hypersetup{hidelinks}%
      \insertshorttitle
    \end{beamercolorbox}%
    \begin{beamercolorbox}[
      wd=.333333\paperwidth,
      ht=2.25ex,
      dp=1ex,
      center
    ]{date in head/foot}%
      \usebeamerfont{date in head/foot}%
      \insertshortdate
      \hspace*{10ex}
      \insertframenumber{} / \inserttotalframenumber\hspace*{1ex}
    \end{beamercolorbox}%
  }
  \par
  \vspace{0pt}%
}

\title{Your Title}
\subtitle{Your (Optional) Subtitle}
\author{Author}
\date{\today}

\begin{document}
$\ldots$
\end{document}\end{lstlisting}
\end{frame}

\section{Beamer}
\begin{frame}[fragile]{Beamer}
  This presentation template is based on Beamer using the Madrid theme.
  \begin{block}{Beamer}
    Beamer is a \LaTeX{} document class for creating slides for presentations.

    OverLeaf has a nice tutorial on Beamer: \url{https://www.overleaf.com/learn/latex/Beamer}.
  \end{block}
  However, the (very brief) TLDR is:
  \begin{itemize}
    \item Use \T{frame} to create slides:
    \begin{lstlisting}[language=MyTex]
\begin$ ${frame}{My Title}
$\ldots$
\end$ ${frame}\end{lstlisting}
    \item Use \T{block} to create blocks within frames: 
    \begin{lstlisting}[language=MyTex]
\begin$ ${block}{My Title}
$\ldots$
\end$ ${block}\end{lstlisting}
  \end{itemize}
\end{frame}

\section{Preambles}
\begin{frame}[fragile]{Articles and Presentations}
  Specify the type of document you are writing by choosing a preamble.
  \begin{block}{Article}
    For writing articles, use the \T{article} preamble at the top of your document:
    \begin{lstlisting}[language=MyTex]
\documentclass{article}
\usepackage{geometry}
\geometry{margin=1in}

% Enable default preamble
% Setup
\usepackage[utf8]{inputenc}

% Math
\usepackage{amsmath, amssymb, mathtools}
\usepackage{xspace} % for making spacing nicer in macros

\newcommand{\xor}{\oplus}
\newcommand{\beq}{\stackrel{?}{=}}
\renewcommand{\mod}{\text{ mod }}
\newcommand{\QED}{\ensuremath{\square}}

\newcommand{\mc}[1]{\ensuremath{\mathcal{#1}}\xspace}
\newcommand{\mb}[1]{\ensuremath{\mathbb{#1}}\xspace}

%% Mathcal Chars
\newcommand{\Ac}{\mc{A}}
\newcommand{\Bc}{\mc{B}}
\newcommand{\Cc}{\mc{C}}
\newcommand{\Dc}{\mc{D}}
\newcommand{\Ec}{\mc{E}}
\newcommand{\Fc}{\mc{F}}
\newcommand{\Gc}{\mc{G}}
\newcommand{\Hc}{\mc{H}}
\newcommand{\Ic}{\mc{I}}
\newcommand{\Jc}{\mc{J}}
\newcommand{\Kc}{\mc{K}}
\newcommand{\Lc}{\mc{L}}
\newcommand{\Mc}{\mc{M}}
\newcommand{\Nc}{\mc{N}}
\newcommand{\Oc}{\mc{O}}
\newcommand{\Pc}{\mc{P}}
\newcommand{\Qc}{\mc{Q}}
\newcommand{\Rc}{\mc{R}}
\newcommand{\Sc}{\mc{S}}
\newcommand{\Tc}{\mc{T}}
\newcommand{\Uc}{\mc{U}}
\newcommand{\Vc}{\mc{V}}
\newcommand{\Wc}{\mc{W}}
\newcommand{\Xc}{\mc{X}}
\newcommand{\Yc}{\mc{Y}}
\newcommand{\Zc}{\mc{Z}}

%% Mathbb Chars
\newcommand{\Ab}{\mb{A}}
\newcommand{\Bb}{\mb{B}}
\newcommand{\Cb}{\mb{C}}
\newcommand{\Db}{\mb{D}}
\newcommand{\Eb}{\mb{E}}
\newcommand{\Fb}{\mb{F}}
\newcommand{\Gb}{\mb{G}}
\newcommand{\Hb}{\mb{H}}
\newcommand{\Ib}{\mb{I}}
\newcommand{\Jb}{\mb{J}}
\newcommand{\Kb}{\mb{K}}
\newcommand{\Lb}{\mb{L}}
\newcommand{\Mb}{\mb{M}}
\newcommand{\Nb}{\mb{N}}
\newcommand{\Ob}{\mb{O}}
\newcommand{\Pb}{\mb{P}}
\newcommand{\Qb}{\mb{Q}}
\newcommand{\Rb}{\mb{R}}
\newcommand{\Sb}{\mb{S}}
\newcommand{\Tb}{\mb{T}}
\newcommand{\Ub}{\mb{U}}
\newcommand{\Vb}{\mb{V}}
\newcommand{\Wb}{\mb{W}}
\newcommand{\Xb}{\mb{X}}
\newcommand{\Yb}{\mb{Y}}
\newcommand{\Zb}{\mb{Z}}

% Plots
\usepackage{pgfplots}
\pgfplotsset{compat = newest}
\NewDocumentCommand{\plot}{O{-2} O{10} O{-2} O{10} m}{
  \begin{center}
    \begin{tikzpicture}
      \begin{axis}[
        xmin=#1, xmax=#2, ymin=#3, ymax=#4, domain=#1:#2,
        xticklabel=\empty, yticklabel=\empty,
        minor tick num=1, axis lines = middle,
        xlabel=$x$, ylabel=$y$,
        label style = {at={(ticklabel cs:1.1)}}]
          \addplot[
              samples = 300,
          ] {#5};
      \end{axis}
    \end{tikzpicture}
  \end{center}
}

% Color
\usepackage{xcolor}

% Images
\usepackage{graphicx}
\newcommand{\img}[2][0.5]{
    \begin{center}
        \includegraphics[scale=#1]{#2}
    \end{center}
}

% Common text commands
\newcommand{\I}[1]{\textit{#1}}
\newcommand{\B}[1]{\textbf{#1}}
\newcommand{\U}[1]{\underline{#1}}
\newcommand{\T}[1]{\texttt{#1}}
\newcommand{\X}[1]{\sout{#1}}

% Special sections
\newcommand{\todo}{\textbf{\color{red} TODO}}
\newcommand{\comment}[1]{\textbf{\color{gray} Comment: #1}}

% Lists
\usepackage{enumitem} % for custom list enumeration

% TikZ
\usetikzlibrary{arrows}
\usetikzlibrary{shapes}
\usetikzlibrary{automata}

% Tables 
\usepackage{float}

% Bibliography
\usepackage[square,numbers]{natbib}
\bibliographystyle{abbrvnat}

% Links 
\usepackage{hyperref}
\hypersetup{
    colorlinks=true,
    linkcolor=black,
    filecolor=magenta,      
    urlcolor=blue,
}

% Email
\newcommand{\email}[2]{
  \href{mailto:#1}{#2}
}

 \end{lstlisting}
  \end{block}
  \begin{block}{Presentation}
    For writing presentations, use the \T{presentation} preamble at the top of your document:
    \begin{lstlisting}[language=MyTex]
\documentclass{beamer}

% To avoid problems with .out files
\usepackage{bookmark}

% Overall theme
\usetheme{Madrid}

% Enable default preamble
% Setup
\usepackage[utf8]{inputenc}

% Math
\usepackage{amsmath, amssymb, mathtools}
\usepackage{xspace} % for making spacing nicer in macros

\newcommand{\xor}{\oplus}
\newcommand{\beq}{\stackrel{?}{=}}
\renewcommand{\mod}{\text{ mod }}
\newcommand{\QED}{\ensuremath{\square}}

\newcommand{\mc}[1]{\ensuremath{\mathcal{#1}}\xspace}
\newcommand{\mb}[1]{\ensuremath{\mathbb{#1}}\xspace}

%% Mathcal Chars
\newcommand{\Ac}{\mc{A}}
\newcommand{\Bc}{\mc{B}}
\newcommand{\Cc}{\mc{C}}
\newcommand{\Dc}{\mc{D}}
\newcommand{\Ec}{\mc{E}}
\newcommand{\Fc}{\mc{F}}
\newcommand{\Gc}{\mc{G}}
\newcommand{\Hc}{\mc{H}}
\newcommand{\Ic}{\mc{I}}
\newcommand{\Jc}{\mc{J}}
\newcommand{\Kc}{\mc{K}}
\newcommand{\Lc}{\mc{L}}
\newcommand{\Mc}{\mc{M}}
\newcommand{\Nc}{\mc{N}}
\newcommand{\Oc}{\mc{O}}
\newcommand{\Pc}{\mc{P}}
\newcommand{\Qc}{\mc{Q}}
\newcommand{\Rc}{\mc{R}}
\newcommand{\Sc}{\mc{S}}
\newcommand{\Tc}{\mc{T}}
\newcommand{\Uc}{\mc{U}}
\newcommand{\Vc}{\mc{V}}
\newcommand{\Wc}{\mc{W}}
\newcommand{\Xc}{\mc{X}}
\newcommand{\Yc}{\mc{Y}}
\newcommand{\Zc}{\mc{Z}}

%% Mathbb Chars
\newcommand{\Ab}{\mb{A}}
\newcommand{\Bb}{\mb{B}}
\newcommand{\Cb}{\mb{C}}
\newcommand{\Db}{\mb{D}}
\newcommand{\Eb}{\mb{E}}
\newcommand{\Fb}{\mb{F}}
\newcommand{\Gb}{\mb{G}}
\newcommand{\Hb}{\mb{H}}
\newcommand{\Ib}{\mb{I}}
\newcommand{\Jb}{\mb{J}}
\newcommand{\Kb}{\mb{K}}
\newcommand{\Lb}{\mb{L}}
\newcommand{\Mb}{\mb{M}}
\newcommand{\Nb}{\mb{N}}
\newcommand{\Ob}{\mb{O}}
\newcommand{\Pb}{\mb{P}}
\newcommand{\Qb}{\mb{Q}}
\newcommand{\Rb}{\mb{R}}
\newcommand{\Sb}{\mb{S}}
\newcommand{\Tb}{\mb{T}}
\newcommand{\Ub}{\mb{U}}
\newcommand{\Vb}{\mb{V}}
\newcommand{\Wb}{\mb{W}}
\newcommand{\Xb}{\mb{X}}
\newcommand{\Yb}{\mb{Y}}
\newcommand{\Zb}{\mb{Z}}

% Plots
\usepackage{pgfplots}
\pgfplotsset{compat = newest}
\NewDocumentCommand{\plot}{O{-2} O{10} O{-2} O{10} m}{
  \begin{center}
    \begin{tikzpicture}
      \begin{axis}[
        xmin=#1, xmax=#2, ymin=#3, ymax=#4, domain=#1:#2,
        xticklabel=\empty, yticklabel=\empty,
        minor tick num=1, axis lines = middle,
        xlabel=$x$, ylabel=$y$,
        label style = {at={(ticklabel cs:1.1)}}]
          \addplot[
              samples = 300,
          ] {#5};
      \end{axis}
    \end{tikzpicture}
  \end{center}
}

% Color
\usepackage{xcolor}

% Images
\usepackage{graphicx}
\newcommand{\img}[2][0.5]{
    \begin{center}
        \includegraphics[scale=#1]{#2}
    \end{center}
}

% Common text commands
\newcommand{\I}[1]{\textit{#1}}
\newcommand{\B}[1]{\textbf{#1}}
\newcommand{\U}[1]{\underline{#1}}
\newcommand{\T}[1]{\texttt{#1}}
\newcommand{\X}[1]{\sout{#1}}

% Special sections
\newcommand{\todo}{\textbf{\color{red} TODO}}
\newcommand{\comment}[1]{\textbf{\color{gray} Comment: #1}}

% Lists
\usepackage{enumitem} % for custom list enumeration

% TikZ
\usetikzlibrary{arrows}
\usetikzlibrary{shapes}
\usetikzlibrary{automata}

% Tables 
\usepackage{float}

% Bibliography
\usepackage[square,numbers]{natbib}
\bibliographystyle{abbrvnat}

% Links 
\usepackage{hyperref}
\hypersetup{
    colorlinks=true,
    linkcolor=black,
    filecolor=magenta,      
    urlcolor=blue,
}

% Email
\newcommand{\email}[2]{
  \href{mailto:#1}{#2}
}

% Make title page simple
\setbeamertemplate{title page}[default]
\setbeamercolor{titlelike}{parent=structure,bg=white}

% Make blocks rectangular
\setbeamertemplate{blocks}[default]

% Make listings look nice
\setlist[itemize]{label= \color{darkgray} $\bullet$}
\setlist[enumerate]{label= \color{darkgray} \arabic*.}

% Fix hyperref interference with beamer footline
\setbeamertemplate{footline}{%
  \leavevmode
  \hbox{%
    \begin{beamercolorbox}[
      wd=.333333\paperwidth,
      ht=2.25ex,
      dp=1ex,
      center
    ]{author in head/foot}%
      \usebeamerfont{author in head/foot}%
      \insertshortauthor
    \end{beamercolorbox}%
    \begin{beamercolorbox}[
      wd=.333333\paperwidth,
      ht=2.25ex,
      dp=1ex,
      center
    ]{title in head/foot}%
      \usebeamerfont{title in head/foot}%
      \hypersetup{hidelinks}%
      \insertshorttitle
    \end{beamercolorbox}%
    \begin{beamercolorbox}[
      wd=.333333\paperwidth,
      ht=2.25ex,
      dp=1ex,
      center
    ]{date in head/foot}%
      \usebeamerfont{date in head/foot}%
      \insertshortdate
      \hspace*{10ex}
      \insertframenumber{} / \inserttotalframenumber\hspace*{1ex}
    \end{beamercolorbox}%
  }
  \par
  \vspace{0pt}%
} \end{lstlisting} 
  \end{block}
  Many more commands are available both for articles and presentations. Please see the article example for more information.
\end{frame}

\iffalse

\subsection{Default}
\begin{frame}{Default}
  All the following commands are enabled by default.
  \begin{block}{Common Text Commands}
    \begin{itemize}
      \item \T{\textbackslash{}I$\{\ldots\}$} for \I{italic} text
      \item \T{\textbackslash{}B$\{\ldots\}$} for \B{bold} text
      \item \T{\textbackslash{}U$\{\ldots\}$} for \U{underlining} text
      \item \T{\textbackslash{}T$\{\ldots\}$} for \T{typewriter} text
      \item \T{\textbackslash{}X$\{\ldots\}$} for \X{strike-through} text
    \end{itemize}
  \end{block}

  \begin{block}{Special Sections}
    \begin{itemize}
      \item \T{\textbackslash{}todo} for inserting \todo
      \item \T{\textbackslash{}comment$\{\ldots\}$} for inserting a comment: \comment{This is a comment.}
    \end{itemize}
  \end{block}
\end{frame}

\begin{frame}{Default Cont.}
  \begin{block}{Colors}
    \begin{itemize}
      \item \T{\textbackslash{}color$\{\ldots\}$} for changing the color of following text: {\color{red} Red text}
    \end{itemize}
    Using \T{xcolor} package.
  \end{block}
  \begin{block}{Links}
    \begin{itemize}
      \item \T{\textbackslash{}url$\{\ldots\}$} for inserting a link: \url{http://aekvi.com}
    \end{itemize}
    Using \T{hyperref} package.
  \end{block}
  \begin{block}{Citing and Bibliography}
    \begin{itemize}
      \item \T{\textbackslash{}bibliography{refs}} for inserting bibliography (assuming \T{refs.bib} exists)
      \item \T{\textbackslash{}cite$\{\ldots\}$} for citing a reference
      \item \T{\textbackslash{}citep$\{\ldots\}$} for citing a reference in parenthesis
    \end{itemize}
    Using \T{natbib} package.
  \end{block}
\end{frame}

\begin{frame}{Default Cont.}
  \begin{block}{Images}
    \begin{itemize}
      \item \T{\textbackslash{}img$[scale]\{src\}$} for inserting a centered image from \T{src} (assuming \T{src} exists) scaled by \T{scale} (optional, default: 0.5):
      
      \img[0.2]{assets/mandelbrot_set}
    \end{itemize}
  \end{block}
  \begin{block}{Aarhus University, CS Department Logo}
    \begin{itemize}
      \item \T{\textbackslash{}csau} for inserting the Aarhus University, Department of Computer Science logo:
      
      \csau
    \end{itemize}
  \end{block}
\end{frame}

\begin{frame}{Default Cont.}
  \begin{block}{Plots}
    \begin{itemize}
      \item \T{\textbackslash{}plot$[xmin][xmax][ymin][ymax]\{f\}$} for inserting a centered plot of \T{f} with the specified bounds (optional, default: $[-2, 10, -2, 10]$): 
      
      $f = sin(x)$
      
      \plot[-2][10][-2][2]{sin(deg(x))}
      
    \end{itemize}
    Using \T{pgfplots} package.
  \end{block}
\end{frame}

\begin{frame}{Default Cont.}
  \begin{block}{TikZ}
    \begin{itemize}
      \item See OverLeafs tutorial on TikZ: \url{https://www.overleaf.com/learn/latex/TikZ_package}
      \item Using TikZ libraries \T{arrows}, \T{shapes} and \T{automata} by default
    \end{itemize}
  \end{block}
  \begin{block}{Math}
    \begin{itemize}
      \item \T{\textbackslash{}Xb} for any symbol $X \in A \ldots Z$ for \T{mathbb} symbols: \Ab, \Bb, \Cb, \ldots
      \item \T{\textbackslash{}Xc} for any symbol $X \in A \ldots Z$ for \T{mathcal} symbols: \Ac, \Bc, \Cc, \ldots
      \item \T{\textbackslash{}xor} for $\xor$
      \item \T{\textbackslash{}beq} for $\beq$ (Boolean equality)
      \item \T{\textbackslash{}mod} for a (nicely behaving) modulo operator: $a \mod b$
      \item \T{\textbackslash{}QED} for \QED
    \end{itemize}
  \end{block}
\end{frame}

\subsection{Code}
\begin{frame}[fragile]{Code}
  The following commands are only available when using the \T{code} preamble.
  \begin{lstlisting}[language=MyTeX]
% Code 
\usepackage{listings}

\definecolor{codegreen}{rgb}{0,0.6,0}
\definecolor{codegray}{rgb}{0.5,0.5,0.5}
\definecolor{codepurple}{rgb}{0.58,0,0.82}
\definecolor{backcolour}{rgb}{0.95,0.95,0.92}

\lstdefinestyle{mystyle}{
    backgroundcolor=\color{backcolour},   
    commentstyle=\color{codegreen},
    keywordstyle=\color{magenta},
    numberstyle=\tiny\color{codegray},
    stringstyle=\color{codepurple},
    basicstyle=\ttfamily\footnotesize,
    breakatwhitespace=false,         
    breaklines=true,                 
    captionpos=b,                    
    keepspaces=true,                 
    numbers=left,                    
    numbersep=5pt,                  
    showspaces=false,                
    showstringspaces=false,
    showtabs=false,                  
    tabsize=2
}

\lstset{
    basicstyle=\ttfamily,
    mathescape,
    style=mystyle
}

% Add keywords to languages
\newcommand{\addkeywords}[2]{
    \lstdefinelanguage{My#1}[]{#1}{
        morekeywords={#2}
    }
}

\addkeywords{TeX}{
    title, 
    subtitle, 
    author, 
    date, 
    begin, 
    maketitle, 
    today, 
    end, 
    documentclass
}
 \end{lstlisting}
  We make use of the \T{lstlisting} environment from the \T{listings} package.
  \begin{lstlisting}[language=MyTeX]
\begin$ ${lstlisting}[language=$\ldots$] 
$\ldots$
\end$ ${lstlisting} \end{lstlisting}
\end{frame}
\subsection{Logic}
\subsection{Protocols}

\fi
\end{document}