\documentclass{article}
\usepackage{geometry}
\geometry{margin=1in}

% Enable default preamble
% Setup
\usepackage[utf8]{inputenc}

% Math
\usepackage{amsmath, amssymb, mathtools}
\usepackage{xspace} % for making spacing nicer in macros

\newcommand{\xor}{\oplus}
\newcommand{\beq}{\stackrel{?}{=}}
\renewcommand{\mod}{\text{ mod }}
\newcommand{\QED}{\ensuremath{\square}}

\newcommand{\mc}[1]{\ensuremath{\mathcal{#1}}\xspace}
\newcommand{\mb}[1]{\ensuremath{\mathbb{#1}}\xspace}

%% Mathcal Chars
\newcommand{\Ac}{\mc{A}}
\newcommand{\Bc}{\mc{B}}
\newcommand{\Cc}{\mc{C}}
\newcommand{\Dc}{\mc{D}}
\newcommand{\Ec}{\mc{E}}
\newcommand{\Fc}{\mc{F}}
\newcommand{\Gc}{\mc{G}}
\newcommand{\Hc}{\mc{H}}
\newcommand{\Ic}{\mc{I}}
\newcommand{\Jc}{\mc{J}}
\newcommand{\Kc}{\mc{K}}
\newcommand{\Lc}{\mc{L}}
\newcommand{\Mc}{\mc{M}}
\newcommand{\Nc}{\mc{N}}
\newcommand{\Oc}{\mc{O}}
\newcommand{\Pc}{\mc{P}}
\newcommand{\Qc}{\mc{Q}}
\newcommand{\Rc}{\mc{R}}
\newcommand{\Sc}{\mc{S}}
\newcommand{\Tc}{\mc{T}}
\newcommand{\Uc}{\mc{U}}
\newcommand{\Vc}{\mc{V}}
\newcommand{\Wc}{\mc{W}}
\newcommand{\Xc}{\mc{X}}
\newcommand{\Yc}{\mc{Y}}
\newcommand{\Zc}{\mc{Z}}

%% Mathbb Chars
\newcommand{\Ab}{\mb{A}}
\newcommand{\Bb}{\mb{B}}
\newcommand{\Cb}{\mb{C}}
\newcommand{\Db}{\mb{D}}
\newcommand{\Eb}{\mb{E}}
\newcommand{\Fb}{\mb{F}}
\newcommand{\Gb}{\mb{G}}
\newcommand{\Hb}{\mb{H}}
\newcommand{\Ib}{\mb{I}}
\newcommand{\Jb}{\mb{J}}
\newcommand{\Kb}{\mb{K}}
\newcommand{\Lb}{\mb{L}}
\newcommand{\Mb}{\mb{M}}
\newcommand{\Nb}{\mb{N}}
\newcommand{\Ob}{\mb{O}}
\newcommand{\Pb}{\mb{P}}
\newcommand{\Qb}{\mb{Q}}
\newcommand{\Rb}{\mb{R}}
\newcommand{\Sb}{\mb{S}}
\newcommand{\Tb}{\mb{T}}
\newcommand{\Ub}{\mb{U}}
\newcommand{\Vb}{\mb{V}}
\newcommand{\Wb}{\mb{W}}
\newcommand{\Xb}{\mb{X}}
\newcommand{\Yb}{\mb{Y}}
\newcommand{\Zb}{\mb{Z}}

% Plots
\usepackage{pgfplots}
\pgfplotsset{compat = newest}
\NewDocumentCommand{\plot}{O{-2} O{10} O{-2} O{10} m}{
  \begin{center}
    \begin{tikzpicture}
      \begin{axis}[
        xmin=#1, xmax=#2, ymin=#3, ymax=#4, domain=#1:#2,
        xticklabel=\empty, yticklabel=\empty,
        minor tick num=1, axis lines = middle,
        xlabel=$x$, ylabel=$y$,
        label style = {at={(ticklabel cs:1.1)}}]
          \addplot[
              samples = 300,
          ] {#5};
      \end{axis}
    \end{tikzpicture}
  \end{center}
}

% Color
\usepackage{xcolor}

% Images
\usepackage{graphicx}
\newcommand{\img}[2][0.5]{
    \begin{center}
        \includegraphics[scale=#1]{#2}
    \end{center}
}

% Common text commands
\newcommand{\I}[1]{\textit{#1}}
\newcommand{\B}[1]{\textbf{#1}}
\newcommand{\U}[1]{\underline{#1}}
\newcommand{\T}[1]{\texttt{#1}}
\newcommand{\X}[1]{\sout{#1}}

% Special sections
\newcommand{\todo}{\textbf{\color{red} TODO}}
\newcommand{\comment}[1]{\textbf{\color{gray} Comment: #1}}

% Lists
\usepackage{enumitem} % for custom list enumeration

% TikZ
\usetikzlibrary{arrows}
\usetikzlibrary{shapes}
\usetikzlibrary{automata}

% Tables 
\usepackage{float}

% Bibliography
\usepackage[square,numbers]{natbib}
\bibliographystyle{abbrvnat}

% Links 
\usepackage{hyperref}
\hypersetup{
    colorlinks=true,
    linkcolor=black,
    filecolor=magenta,      
    urlcolor=blue,
}

% Email
\newcommand{\email}[2]{
  \href{mailto:#1}{#2}
}


% Questions 
\newcommand{\question}[1]{%
    \vspace*{6pt}
    \setlength{\fboxsep}{0.15cm}
    \noindent
    \colorbox{gray!15}{%
        \begin{minipage}{1\linewidth}%
            \vspace*{2pt}
            \textbf{Question: }#1
            \vspace*{2pt}
        \end{minipage}%
    }
    \setlength{\parindent}{0.5cm}
    \vspace*{6pt}
    \\
    \noindent
}

% Code 
\usepackage{listings}

\definecolor{codegreen}{rgb}{0,0.6,0}
\definecolor{codegray}{rgb}{0.5,0.5,0.5}
\definecolor{codepurple}{rgb}{0.58,0,0.82}
\definecolor{backcolour}{rgb}{0.95,0.95,0.92}

\lstdefinestyle{mystyle}{
    backgroundcolor=\color{backcolour},   
    commentstyle=\color{codegreen},
    keywordstyle=\color{magenta},
    numberstyle=\tiny\color{codegray},
    stringstyle=\color{codepurple},
    basicstyle=\ttfamily\footnotesize,
    breakatwhitespace=false,         
    breaklines=true,                 
    captionpos=b,                    
    keepspaces=true,                 
    numbers=left,                    
    numbersep=5pt,                  
    showspaces=false,                
    showstringspaces=false,
    showtabs=false,                  
    tabsize=2
}

\lstset{
    basicstyle=\ttfamily,
    mathescape,
    style=mystyle
}

% Add keywords to languages
\newcommand{\addkeywords}[2]{
    \lstdefinelanguage{My#1}[]{#1}{
        morekeywords={#2}
    }
}

\addkeywords{TeX}{
    title, 
    subtitle, 
    author, 
    date, 
    begin, 
    maketitle, 
    today, 
    end, 
    documentclass
}

% LTL, CTL and CTL*
\newcommand{\eventually}{\ensuremath{\Diamond}}
\newcommand{\always}{\ensuremath{\Box}}
\newcommand{\until}{\text{ U }}
\newcommand{\release}{\text{ R }}
\newcommand{\weakuntil}{\text{ W }}
\newcommand{\nex}{\ensuremath{\bigcirc}}
\newcommand{\true}{\text{true}}
\newcommand{\false}{\text{false}}

% Similarity and Bisimilarity
\newcommand{\simulatedby}{\ensuremath{\preceq}}
\newcommand{\similar}{\ensuremath{\simeq}}
\newcommand{\bisimilar}{\ensuremath{\sim}}

% Crytpographic protocols
\usepackage{msc}
\usepackage[normalem]{ulem}

% fix msc package not being able to display math characters
\makeatletter
\def\msc@selfmess#1#2#3{
  \IfStrEq{\mscget{label position}}{left}{\xdef\msc@tempc{-1}}{\xdef\msc@tempc{1}}
  \xdef\msc@tempb{[\mscget{label position},/msc,message,message loop={#3},
            \ifmsc@isstar replay,\fi, inner sep=0pt,
           /tikz/pos=\mscget{pos}, every message, \msc@options]
           (#1)
              to node[xshift=\msc@tempc*\mscget{label distance}](msc@lastnode){\unexpanded\expandafter{\msc@mess@name}}
           (#2);
          }
}

\newcommand{\protocol}[3][]{
  \begin{center}
    \drawframe{no}
    \begin{msc}[msc keyword=, instance distance=15em, environment distance=10em, head top distance=5em, label distance=0.2em]{Protocol: \textbf{#2}\\\ \\ \hfill Public: #1 \hfill \mbox{}}
      #3
    \end{msc}
  \end{center}
}
\newcommand{\party}[2]{\declinst[]{#1}{}{#2}}
\newcommand{\knows}[3]{
  \mess[label position=#3, side=#3]{#2}{#1}{#1}[0]
  \nextlevel
}
\newcommand{\cond}[2]{
  \condition{#1}{#2}
  \nextlevel[2]
}
\newcommand{\msg}[3]{
  \mess{#2}{#1}{#3}
  \nextlevel
}


\usepackage{graphicx}
\usepackage{array}
\newcolumntype{M}[1]{>{\arraybackslash}m{#1}}
\DeclareDocumentCommand{\feature}{mmmO{}}{
  \\ \hline
  #1 & \vspace*{1.05em}\begin{lstlisting}[language=MyTeX, numbers=none, backgroundcolor=\color{white}, linewidth=2.5cm]^^J
$\backslash{}$#2$\{\ldots\}$^^J 
        \end{lstlisting} &
  \vspace*{1.05em}\begin{lstlisting}[language=MyTeX, numbers=none, backgroundcolor=\color{white}, linewidth=5cm]^^J
$\backslash{}$#2$\{$#3$\}$ #4^^J 
    \end{lstlisting} &
  \csname#2\endcsname{#3}#4
}
\newcommand{\customfeature}[4]{
  \\ \hline
  #1 & \vspace*{1.05em}\begin{lstlisting}[language=MyTeX, numbers=none, backgroundcolor=\color{white}, linewidth=2.5cm]^^J
#2^^J 
        \end{lstlisting} &
  \vspace*{1.05em}\begin{lstlisting}[language=MyTeX, numbers=none, backgroundcolor=\color{white}, linewidth=5cm]^^J
#3^^J 
    \end{lstlisting} &
  #4
}
\newcommand{\block}[2]{
  \subsubsection{#1}
  \begin{center}
    \begin{tabular}{M{3cm} M{2.5cm} M{5cm} M{4cm}}
      & \B{Usage} & \B{Example} & \B{Result} 
     #2 \\
     \hline 
    \end{tabular}
  \end{center}
}

\title{Preambles in \LaTeX}
\author{Anders B. Clausen\\ \small \url{https://github.com/anbclausen}}
\date{\today}

\begin{document}
\maketitle

\tableofcontents

\section{Setup}
Here is how to get going with an article:
\begin{lstlisting}[language=MyTeX]
\documentclass{article}
\usepackage{geometry}
\geometry{margin=1in}

% Enable default preamble
% Setup
\usepackage[utf8]{inputenc}

% Math
\usepackage{amsmath, amssymb, mathtools}
\usepackage{xspace} % for making spacing nicer in macros

\newcommand{\xor}{\oplus}
\newcommand{\beq}{\stackrel{?}{=}}
\renewcommand{\mod}{\text{ mod }}
\newcommand{\QED}{\ensuremath{\square}}

\newcommand{\mc}[1]{\ensuremath{\mathcal{#1}}\xspace}
\newcommand{\mb}[1]{\ensuremath{\mathbb{#1}}\xspace}

%% Mathcal Chars
\newcommand{\Ac}{\mc{A}}
\newcommand{\Bc}{\mc{B}}
\newcommand{\Cc}{\mc{C}}
\newcommand{\Dc}{\mc{D}}
\newcommand{\Ec}{\mc{E}}
\newcommand{\Fc}{\mc{F}}
\newcommand{\Gc}{\mc{G}}
\newcommand{\Hc}{\mc{H}}
\newcommand{\Ic}{\mc{I}}
\newcommand{\Jc}{\mc{J}}
\newcommand{\Kc}{\mc{K}}
\newcommand{\Lc}{\mc{L}}
\newcommand{\Mc}{\mc{M}}
\newcommand{\Nc}{\mc{N}}
\newcommand{\Oc}{\mc{O}}
\newcommand{\Pc}{\mc{P}}
\newcommand{\Qc}{\mc{Q}}
\newcommand{\Rc}{\mc{R}}
\newcommand{\Sc}{\mc{S}}
\newcommand{\Tc}{\mc{T}}
\newcommand{\Uc}{\mc{U}}
\newcommand{\Vc}{\mc{V}}
\newcommand{\Wc}{\mc{W}}
\newcommand{\Xc}{\mc{X}}
\newcommand{\Yc}{\mc{Y}}
\newcommand{\Zc}{\mc{Z}}

%% Mathbb Chars
\newcommand{\Ab}{\mb{A}}
\newcommand{\Bb}{\mb{B}}
\newcommand{\Cb}{\mb{C}}
\newcommand{\Db}{\mb{D}}
\newcommand{\Eb}{\mb{E}}
\newcommand{\Fb}{\mb{F}}
\newcommand{\Gb}{\mb{G}}
\newcommand{\Hb}{\mb{H}}
\newcommand{\Ib}{\mb{I}}
\newcommand{\Jb}{\mb{J}}
\newcommand{\Kb}{\mb{K}}
\newcommand{\Lb}{\mb{L}}
\newcommand{\Mb}{\mb{M}}
\newcommand{\Nb}{\mb{N}}
\newcommand{\Ob}{\mb{O}}
\newcommand{\Pb}{\mb{P}}
\newcommand{\Qb}{\mb{Q}}
\newcommand{\Rb}{\mb{R}}
\newcommand{\Sb}{\mb{S}}
\newcommand{\Tb}{\mb{T}}
\newcommand{\Ub}{\mb{U}}
\newcommand{\Vb}{\mb{V}}
\newcommand{\Wb}{\mb{W}}
\newcommand{\Xb}{\mb{X}}
\newcommand{\Yb}{\mb{Y}}
\newcommand{\Zb}{\mb{Z}}

% Plots
\usepackage{pgfplots}
\pgfplotsset{compat = newest}
\NewDocumentCommand{\plot}{O{-2} O{10} O{-2} O{10} m}{
  \begin{center}
    \begin{tikzpicture}
      \begin{axis}[
        xmin=#1, xmax=#2, ymin=#3, ymax=#4, domain=#1:#2,
        xticklabel=\empty, yticklabel=\empty,
        minor tick num=1, axis lines = middle,
        xlabel=$x$, ylabel=$y$,
        label style = {at={(ticklabel cs:1.1)}}]
          \addplot[
              samples = 300,
          ] {#5};
      \end{axis}
    \end{tikzpicture}
  \end{center}
}

% Color
\usepackage{xcolor}

% Images
\usepackage{graphicx}
\newcommand{\img}[2][0.5]{
    \begin{center}
        \includegraphics[scale=#1]{#2}
    \end{center}
}

% Common text commands
\newcommand{\I}[1]{\textit{#1}}
\newcommand{\B}[1]{\textbf{#1}}
\newcommand{\U}[1]{\underline{#1}}
\newcommand{\T}[1]{\texttt{#1}}
\newcommand{\X}[1]{\sout{#1}}

% Special sections
\newcommand{\todo}{\textbf{\color{red} TODO}}
\newcommand{\comment}[1]{\textbf{\color{gray} Comment: #1}}

% Lists
\usepackage{enumitem} % for custom list enumeration

% TikZ
\usetikzlibrary{arrows}
\usetikzlibrary{shapes}
\usetikzlibrary{automata}

% Tables 
\usepackage{float}

% Bibliography
\usepackage[square,numbers]{natbib}
\bibliographystyle{abbrvnat}

% Links 
\usepackage{hyperref}
\hypersetup{
    colorlinks=true,
    linkcolor=black,
    filecolor=magenta,      
    urlcolor=blue,
}

% Email
\newcommand{\email}[2]{
  \href{mailto:#1}{#2}
}



\title{Your Title}
\author{Author}
\date{\today}

\begin{document}
\maketitle
$\ldots$
\end{document}\end{lstlisting}

\section{Preambles}
\subsection{Articles and Presentations}
Specify the type of document you are writing by choosing a preamble. For writing articles, use the \T{article} preamble at the top of your document:
\begin{lstlisting}[language=MyTex]
\documentclass{article}
\usepackage{geometry}
\geometry{margin=1in}

% Enable default preamble
% Setup
\usepackage[utf8]{inputenc}

% Math
\usepackage{amsmath, amssymb, mathtools}
\usepackage{xspace} % for making spacing nicer in macros

\newcommand{\xor}{\oplus}
\newcommand{\beq}{\stackrel{?}{=}}
\renewcommand{\mod}{\text{ mod }}
\newcommand{\QED}{\ensuremath{\square}}

\newcommand{\mc}[1]{\ensuremath{\mathcal{#1}}\xspace}
\newcommand{\mb}[1]{\ensuremath{\mathbb{#1}}\xspace}

%% Mathcal Chars
\newcommand{\Ac}{\mc{A}}
\newcommand{\Bc}{\mc{B}}
\newcommand{\Cc}{\mc{C}}
\newcommand{\Dc}{\mc{D}}
\newcommand{\Ec}{\mc{E}}
\newcommand{\Fc}{\mc{F}}
\newcommand{\Gc}{\mc{G}}
\newcommand{\Hc}{\mc{H}}
\newcommand{\Ic}{\mc{I}}
\newcommand{\Jc}{\mc{J}}
\newcommand{\Kc}{\mc{K}}
\newcommand{\Lc}{\mc{L}}
\newcommand{\Mc}{\mc{M}}
\newcommand{\Nc}{\mc{N}}
\newcommand{\Oc}{\mc{O}}
\newcommand{\Pc}{\mc{P}}
\newcommand{\Qc}{\mc{Q}}
\newcommand{\Rc}{\mc{R}}
\newcommand{\Sc}{\mc{S}}
\newcommand{\Tc}{\mc{T}}
\newcommand{\Uc}{\mc{U}}
\newcommand{\Vc}{\mc{V}}
\newcommand{\Wc}{\mc{W}}
\newcommand{\Xc}{\mc{X}}
\newcommand{\Yc}{\mc{Y}}
\newcommand{\Zc}{\mc{Z}}

%% Mathbb Chars
\newcommand{\Ab}{\mb{A}}
\newcommand{\Bb}{\mb{B}}
\newcommand{\Cb}{\mb{C}}
\newcommand{\Db}{\mb{D}}
\newcommand{\Eb}{\mb{E}}
\newcommand{\Fb}{\mb{F}}
\newcommand{\Gb}{\mb{G}}
\newcommand{\Hb}{\mb{H}}
\newcommand{\Ib}{\mb{I}}
\newcommand{\Jb}{\mb{J}}
\newcommand{\Kb}{\mb{K}}
\newcommand{\Lb}{\mb{L}}
\newcommand{\Mb}{\mb{M}}
\newcommand{\Nb}{\mb{N}}
\newcommand{\Ob}{\mb{O}}
\newcommand{\Pb}{\mb{P}}
\newcommand{\Qb}{\mb{Q}}
\newcommand{\Rb}{\mb{R}}
\newcommand{\Sb}{\mb{S}}
\newcommand{\Tb}{\mb{T}}
\newcommand{\Ub}{\mb{U}}
\newcommand{\Vb}{\mb{V}}
\newcommand{\Wb}{\mb{W}}
\newcommand{\Xb}{\mb{X}}
\newcommand{\Yb}{\mb{Y}}
\newcommand{\Zb}{\mb{Z}}

% Plots
\usepackage{pgfplots}
\pgfplotsset{compat = newest}
\NewDocumentCommand{\plot}{O{-2} O{10} O{-2} O{10} m}{
  \begin{center}
    \begin{tikzpicture}
      \begin{axis}[
        xmin=#1, xmax=#2, ymin=#3, ymax=#4, domain=#1:#2,
        xticklabel=\empty, yticklabel=\empty,
        minor tick num=1, axis lines = middle,
        xlabel=$x$, ylabel=$y$,
        label style = {at={(ticklabel cs:1.1)}}]
          \addplot[
              samples = 300,
          ] {#5};
      \end{axis}
    \end{tikzpicture}
  \end{center}
}

% Color
\usepackage{xcolor}

% Images
\usepackage{graphicx}
\newcommand{\img}[2][0.5]{
    \begin{center}
        \includegraphics[scale=#1]{#2}
    \end{center}
}

% Common text commands
\newcommand{\I}[1]{\textit{#1}}
\newcommand{\B}[1]{\textbf{#1}}
\newcommand{\U}[1]{\underline{#1}}
\newcommand{\T}[1]{\texttt{#1}}
\newcommand{\X}[1]{\sout{#1}}

% Special sections
\newcommand{\todo}{\textbf{\color{red} TODO}}
\newcommand{\comment}[1]{\textbf{\color{gray} Comment: #1}}

% Lists
\usepackage{enumitem} % for custom list enumeration

% TikZ
\usetikzlibrary{arrows}
\usetikzlibrary{shapes}
\usetikzlibrary{automata}

% Tables 
\usepackage{float}

% Bibliography
\usepackage[square,numbers]{natbib}
\bibliographystyle{abbrvnat}

% Links 
\usepackage{hyperref}
\hypersetup{
    colorlinks=true,
    linkcolor=black,
    filecolor=magenta,      
    urlcolor=blue,
}

% Email
\newcommand{\email}[2]{
  \href{mailto:#1}{#2}
}

 \end{lstlisting}
For writing presentations, use the \T{presentation} preamble at the top of your document:
\begin{lstlisting}[language=MyTex]
\documentclass{beamer}

% To avoid problems with .out files
\usepackage{bookmark}

% Overall theme
\usetheme{Madrid}

% Enable default preamble
% Setup
\usepackage[utf8]{inputenc}

% Math
\usepackage{amsmath, amssymb, mathtools}
\usepackage{xspace} % for making spacing nicer in macros

\newcommand{\xor}{\oplus}
\newcommand{\beq}{\stackrel{?}{=}}
\renewcommand{\mod}{\text{ mod }}
\newcommand{\QED}{\ensuremath{\square}}

\newcommand{\mc}[1]{\ensuremath{\mathcal{#1}}\xspace}
\newcommand{\mb}[1]{\ensuremath{\mathbb{#1}}\xspace}

%% Mathcal Chars
\newcommand{\Ac}{\mc{A}}
\newcommand{\Bc}{\mc{B}}
\newcommand{\Cc}{\mc{C}}
\newcommand{\Dc}{\mc{D}}
\newcommand{\Ec}{\mc{E}}
\newcommand{\Fc}{\mc{F}}
\newcommand{\Gc}{\mc{G}}
\newcommand{\Hc}{\mc{H}}
\newcommand{\Ic}{\mc{I}}
\newcommand{\Jc}{\mc{J}}
\newcommand{\Kc}{\mc{K}}
\newcommand{\Lc}{\mc{L}}
\newcommand{\Mc}{\mc{M}}
\newcommand{\Nc}{\mc{N}}
\newcommand{\Oc}{\mc{O}}
\newcommand{\Pc}{\mc{P}}
\newcommand{\Qc}{\mc{Q}}
\newcommand{\Rc}{\mc{R}}
\newcommand{\Sc}{\mc{S}}
\newcommand{\Tc}{\mc{T}}
\newcommand{\Uc}{\mc{U}}
\newcommand{\Vc}{\mc{V}}
\newcommand{\Wc}{\mc{W}}
\newcommand{\Xc}{\mc{X}}
\newcommand{\Yc}{\mc{Y}}
\newcommand{\Zc}{\mc{Z}}

%% Mathbb Chars
\newcommand{\Ab}{\mb{A}}
\newcommand{\Bb}{\mb{B}}
\newcommand{\Cb}{\mb{C}}
\newcommand{\Db}{\mb{D}}
\newcommand{\Eb}{\mb{E}}
\newcommand{\Fb}{\mb{F}}
\newcommand{\Gb}{\mb{G}}
\newcommand{\Hb}{\mb{H}}
\newcommand{\Ib}{\mb{I}}
\newcommand{\Jb}{\mb{J}}
\newcommand{\Kb}{\mb{K}}
\newcommand{\Lb}{\mb{L}}
\newcommand{\Mb}{\mb{M}}
\newcommand{\Nb}{\mb{N}}
\newcommand{\Ob}{\mb{O}}
\newcommand{\Pb}{\mb{P}}
\newcommand{\Qb}{\mb{Q}}
\newcommand{\Rb}{\mb{R}}
\newcommand{\Sb}{\mb{S}}
\newcommand{\Tb}{\mb{T}}
\newcommand{\Ub}{\mb{U}}
\newcommand{\Vb}{\mb{V}}
\newcommand{\Wb}{\mb{W}}
\newcommand{\Xb}{\mb{X}}
\newcommand{\Yb}{\mb{Y}}
\newcommand{\Zb}{\mb{Z}}

% Plots
\usepackage{pgfplots}
\pgfplotsset{compat = newest}
\NewDocumentCommand{\plot}{O{-2} O{10} O{-2} O{10} m}{
  \begin{center}
    \begin{tikzpicture}
      \begin{axis}[
        xmin=#1, xmax=#2, ymin=#3, ymax=#4, domain=#1:#2,
        xticklabel=\empty, yticklabel=\empty,
        minor tick num=1, axis lines = middle,
        xlabel=$x$, ylabel=$y$,
        label style = {at={(ticklabel cs:1.1)}}]
          \addplot[
              samples = 300,
          ] {#5};
      \end{axis}
    \end{tikzpicture}
  \end{center}
}

% Color
\usepackage{xcolor}

% Images
\usepackage{graphicx}
\newcommand{\img}[2][0.5]{
    \begin{center}
        \includegraphics[scale=#1]{#2}
    \end{center}
}

% Common text commands
\newcommand{\I}[1]{\textit{#1}}
\newcommand{\B}[1]{\textbf{#1}}
\newcommand{\U}[1]{\underline{#1}}
\newcommand{\T}[1]{\texttt{#1}}
\newcommand{\X}[1]{\sout{#1}}

% Special sections
\newcommand{\todo}{\textbf{\color{red} TODO}}
\newcommand{\comment}[1]{\textbf{\color{gray} Comment: #1}}

% Lists
\usepackage{enumitem} % for custom list enumeration

% TikZ
\usetikzlibrary{arrows}
\usetikzlibrary{shapes}
\usetikzlibrary{automata}

% Tables 
\usepackage{float}

% Bibliography
\usepackage[square,numbers]{natbib}
\bibliographystyle{abbrvnat}

% Links 
\usepackage{hyperref}
\hypersetup{
    colorlinks=true,
    linkcolor=black,
    filecolor=magenta,      
    urlcolor=blue,
}

% Email
\newcommand{\email}[2]{
  \href{mailto:#1}{#2}
}

% Make title page simple
\setbeamertemplate{title page}[default]
\setbeamercolor{titlelike}{parent=structure,bg=white}

% Make blocks rectangular
\setbeamertemplate{blocks}[default]

% Make listings look nice
\setlist[itemize]{label= \color{darkgray} $\bullet$}
\setlist[enumerate]{label= \color{darkgray} \arabic*.}

% Fix hyperref interference with beamer footline
\setbeamertemplate{footline}{%
  \leavevmode
  \hbox{%
    \begin{beamercolorbox}[
      wd=.333333\paperwidth,
      ht=2.25ex,
      dp=1ex,
      center
    ]{author in head/foot}%
      \usebeamerfont{author in head/foot}%
      \insertshortauthor
    \end{beamercolorbox}%
    \begin{beamercolorbox}[
      wd=.333333\paperwidth,
      ht=2.25ex,
      dp=1ex,
      center
    ]{title in head/foot}%
      \usebeamerfont{title in head/foot}%
      \hypersetup{hidelinks}%
      \insertshorttitle
    \end{beamercolorbox}%
    \begin{beamercolorbox}[
      wd=.333333\paperwidth,
      ht=2.25ex,
      dp=1ex,
      center
    ]{date in head/foot}%
      \usebeamerfont{date in head/foot}%
      \insertshortdate
      \hspace*{10ex}
      \insertframenumber{} / \inserttotalframenumber\hspace*{1ex}
    \end{beamercolorbox}%
  }
  \par
  \vspace{0pt}%
} \end{lstlisting} 

\subsection{Default}
The preambles for articles and presentations offer many kinds of features. Here are all the ones that are enabled by default! If you do not want to use the article or presentation preamble, you can use the \T{default} preamble at the top of your document:
\begin{lstlisting}[language=MyTex]
% Setup
\usepackage[utf8]{inputenc}

% Math
\usepackage{amsmath, amssymb, mathtools}
\usepackage{xspace} % for making spacing nicer in macros

\newcommand{\xor}{\oplus}
\newcommand{\beq}{\stackrel{?}{=}}
\renewcommand{\mod}{\text{ mod }}
\newcommand{\QED}{\ensuremath{\square}}

\newcommand{\mc}[1]{\ensuremath{\mathcal{#1}}\xspace}
\newcommand{\mb}[1]{\ensuremath{\mathbb{#1}}\xspace}

%% Mathcal Chars
\newcommand{\Ac}{\mc{A}}
\newcommand{\Bc}{\mc{B}}
\newcommand{\Cc}{\mc{C}}
\newcommand{\Dc}{\mc{D}}
\newcommand{\Ec}{\mc{E}}
\newcommand{\Fc}{\mc{F}}
\newcommand{\Gc}{\mc{G}}
\newcommand{\Hc}{\mc{H}}
\newcommand{\Ic}{\mc{I}}
\newcommand{\Jc}{\mc{J}}
\newcommand{\Kc}{\mc{K}}
\newcommand{\Lc}{\mc{L}}
\newcommand{\Mc}{\mc{M}}
\newcommand{\Nc}{\mc{N}}
\newcommand{\Oc}{\mc{O}}
\newcommand{\Pc}{\mc{P}}
\newcommand{\Qc}{\mc{Q}}
\newcommand{\Rc}{\mc{R}}
\newcommand{\Sc}{\mc{S}}
\newcommand{\Tc}{\mc{T}}
\newcommand{\Uc}{\mc{U}}
\newcommand{\Vc}{\mc{V}}
\newcommand{\Wc}{\mc{W}}
\newcommand{\Xc}{\mc{X}}
\newcommand{\Yc}{\mc{Y}}
\newcommand{\Zc}{\mc{Z}}

%% Mathbb Chars
\newcommand{\Ab}{\mb{A}}
\newcommand{\Bb}{\mb{B}}
\newcommand{\Cb}{\mb{C}}
\newcommand{\Db}{\mb{D}}
\newcommand{\Eb}{\mb{E}}
\newcommand{\Fb}{\mb{F}}
\newcommand{\Gb}{\mb{G}}
\newcommand{\Hb}{\mb{H}}
\newcommand{\Ib}{\mb{I}}
\newcommand{\Jb}{\mb{J}}
\newcommand{\Kb}{\mb{K}}
\newcommand{\Lb}{\mb{L}}
\newcommand{\Mb}{\mb{M}}
\newcommand{\Nb}{\mb{N}}
\newcommand{\Ob}{\mb{O}}
\newcommand{\Pb}{\mb{P}}
\newcommand{\Qb}{\mb{Q}}
\newcommand{\Rb}{\mb{R}}
\newcommand{\Sb}{\mb{S}}
\newcommand{\Tb}{\mb{T}}
\newcommand{\Ub}{\mb{U}}
\newcommand{\Vb}{\mb{V}}
\newcommand{\Wb}{\mb{W}}
\newcommand{\Xb}{\mb{X}}
\newcommand{\Yb}{\mb{Y}}
\newcommand{\Zb}{\mb{Z}}

% Plots
\usepackage{pgfplots}
\pgfplotsset{compat = newest}
\NewDocumentCommand{\plot}{O{-2} O{10} O{-2} O{10} m}{
  \begin{center}
    \begin{tikzpicture}
      \begin{axis}[
        xmin=#1, xmax=#2, ymin=#3, ymax=#4, domain=#1:#2,
        xticklabel=\empty, yticklabel=\empty,
        minor tick num=1, axis lines = middle,
        xlabel=$x$, ylabel=$y$,
        label style = {at={(ticklabel cs:1.1)}}]
          \addplot[
              samples = 300,
          ] {#5};
      \end{axis}
    \end{tikzpicture}
  \end{center}
}

% Color
\usepackage{xcolor}

% Images
\usepackage{graphicx}
\newcommand{\img}[2][0.5]{
    \begin{center}
        \includegraphics[scale=#1]{#2}
    \end{center}
}

% Common text commands
\newcommand{\I}[1]{\textit{#1}}
\newcommand{\B}[1]{\textbf{#1}}
\newcommand{\U}[1]{\underline{#1}}
\newcommand{\T}[1]{\texttt{#1}}
\newcommand{\X}[1]{\sout{#1}}

% Special sections
\newcommand{\todo}{\textbf{\color{red} TODO}}
\newcommand{\comment}[1]{\textbf{\color{gray} Comment: #1}}

% Lists
\usepackage{enumitem} % for custom list enumeration

% TikZ
\usetikzlibrary{arrows}
\usetikzlibrary{shapes}
\usetikzlibrary{automata}

% Tables 
\usepackage{float}

% Bibliography
\usepackage[square,numbers]{natbib}
\bibliographystyle{abbrvnat}

% Links 
\usepackage{hyperref}
\hypersetup{
    colorlinks=true,
    linkcolor=black,
    filecolor=magenta,      
    urlcolor=blue,
}

% Email
\newcommand{\email}[2]{
  \href{mailto:#1}{#2}
} \end{lstlisting}

\block{Common Text Commands}{
  \feature{Italic Text}{I}{This is italic.}
  \feature{Bold Text}{B}{This is bold.}
  \feature{Underlined Text}{U}{This is underlined.}
  \feature{Teletype (Monospace) Text}{T}{This is teletype.}
  \feature{Striked Out Text}{X}{This is striked out.}
}

\block{Special Sections}{
  \customfeature{Todo Marker}{$\backslash{}$todo}{$\backslash{}$todo}{\todo}
  \feature{Comment}{comment}{This is a comment.}
}

\block{Colors}{
  \feature{Colors from \T{xcolor} package}{color}{blue}[I'm blue... Da ba dee da ba daa...]
}

\block{Links and Email Addresses}{  
  \customfeature{Hyper References from \T{hyperref} package}{$\backslash{}$href$\{\ldots\}\{\ldots\}$}{$\backslash{}$href$\{$https://aekvi.com$\}\{$aekvi$\}$}{\href{https://aekvi.com}{aekvi}}

  \customfeature{Emails}{$\backslash{}$email$\{\ldots\}\{\ldots\}$}{$\backslash{}$email$\{$abc@aekvi.com$\}\{$abc$\}$}{\email{abc@aekvi.com}{abc}}
}

\block{Citing and Bibliography}{
  \customfeature{Bibliography}{$\backslash{}$bibliography$\{\ldots\}$}{$\backslash{}$bibliography$\{$refs$\}$}{Renders bibliography assuming \T{refs.bib} file exists. See an example at the end of this document.}
  \feature{Cite}{cite}{clausen}
}
All citation features are from the \T{natbib} package. Learn more about it \href{https://www.overleaf.com/learn/latex/Bibliography_management_with_natbib}{here}.

\block{Images}{
  \customfeature{Images}{$\backslash{}$img$[$scale$]\{\ldots\}$}{$\backslash{}$img$[$0.2$]\{$assets/mandelbrot_set$\}$}{\img[0.2]{assets/mandelbrot_set}assuming assets/mandelbrot\_set.png exists. \T{scale} is optional.}
}

\block{Plots}{
  \customfeature{Plots}{$\backslash{}$plot[xmin]\ \ \ [xmax]\ \ \ \ [ymin]\ \ \ \ [ymax]$\{$f$\}$}{$\backslash{}$plot[-2][10][-2][2]$\qquad\qquad\ \{$sin(deg(x))$\}$}{\begin{center}\scalebox{.6}{\parbox{5cm}{\plot[-2][10][-2][2]{sin(deg(x))}}}\end{center}\T{xmin, xmax, ymin, ymax} are optional.}
}

\subsubsection{TikZ}
See OverLeafs tutorial on TikZ \href{https://www.overleaf.com/learn/latex/TikZ_package}{here}. Using TikZ libraries \T{arrows}, \T{shapes} and \T{automata} by default.

\block{Math}{
  \customfeature{\T{mathbb} symbols}{$\backslash$Xb for $X \in A \ldots Z$}{$\backslash$Ab, $\backslash$Bb, $\backslash$Cb, $\backslash$Db}{\Ab, \Bb, \Cb, \Db}
  \customfeature{\T{mathcal} symbols}{$\backslash$Xc for $X \in A \ldots Z$}{$\backslash$Ac, $\backslash$Bc, $\backslash$Cc, $\backslash$Dc}{\Ac, \Bc, \Cc, \Dc}
  \customfeature{XOR}{$\$\backslash$xor$\$$}{$\$\backslash$xor$\$$}{$\xor$}
  \customfeature{Boolean Equality}{$\$\backslash$beq$\$$}{$\$$a$\ \backslash$beq b$\$$}{$a \beq b$}
  \customfeature{(a nicely behaving) Modulo}{$\$\backslash$mod$\$$}{$\$$a$\ \backslash$mod b$\$$}{$a \mod b$}
  \customfeature{Q.E.D.}{$\backslash$QED}{$\backslash$QED}{\QED}
}

\subsection{Handin}
The handin preamble contains useful features for writing handins. Use the \T{handin} preamble at the top of your document:
\begin{lstlisting}[language=MyTex]
% Questions 
\newcommand{\question}[1]{%
    \vspace*{6pt}
    \setlength{\fboxsep}{0.15cm}
    \noindent
    \colorbox{gray!15}{%
        \begin{minipage}{1\linewidth}%
            \vspace*{2pt}
            \textbf{Question: }#1
            \vspace*{2pt}
        \end{minipage}%
    }
    \setlength{\parindent}{0.5cm}
    \vspace*{6pt}
    \\
    \noindent
}
 \end{lstlisting}

\subsubsection{Question Areas}
Question areas are used to highlight exercise descriptions and solutions. You can define a question like so:
\begin{lstlisting}[language=MyTex]
\question{Prove $\$$P = NP$\$$.}\end{lstlisting}
This will render a question area like this:

\question{Prove $P = NP$.}
\subsection{Code}
The code preamble contains useful features for writing code. Use the \T{code} preamble at the top of your document:
\begin{lstlisting}[language=MyTex]
% Code 
\usepackage{listings}

\definecolor{codegreen}{rgb}{0,0.6,0}
\definecolor{codegray}{rgb}{0.5,0.5,0.5}
\definecolor{codepurple}{rgb}{0.58,0,0.82}
\definecolor{backcolour}{rgb}{0.95,0.95,0.92}

\lstdefinestyle{mystyle}{
    backgroundcolor=\color{backcolour},   
    commentstyle=\color{codegreen},
    keywordstyle=\color{magenta},
    numberstyle=\tiny\color{codegray},
    stringstyle=\color{codepurple},
    basicstyle=\ttfamily\footnotesize,
    breakatwhitespace=false,         
    breaklines=true,                 
    captionpos=b,                    
    keepspaces=true,                 
    numbers=left,                    
    numbersep=5pt,                  
    showspaces=false,                
    showstringspaces=false,
    showtabs=false,                  
    tabsize=2
}

\lstset{
    basicstyle=\ttfamily,
    mathescape,
    style=mystyle
}

% Add keywords to languages
\newcommand{\addkeywords}[2]{
    \lstdefinelanguage{My#1}[]{#1}{
        morekeywords={#2}
    }
}

\addkeywords{TeX}{
    title, 
    subtitle, 
    author, 
    date, 
    begin, 
    maketitle, 
    today, 
    end, 
    documentclass
}
 \end{lstlisting}

\subsubsection{Code Blocks}
The \T{listings} package is used to render code blocks. One can define a code block like so:
\begin{lstlisting}[language=MyTex]
$\backslash$begin{lstlisting}[language=Haskell]
  main = putStrLn "Hello World!"
$\backslash$end{lstlisting}
\end{lstlisting}
This will render a code block like this:
\begin{lstlisting}[language=Haskell]
main = putStrLn "Hello World!"
\end{lstlisting}
Note that everything inside dollar symbols (\$) is rendered as math. This means that you can use math symbols in your code blocks. For example, the following code block:
\begin{lstlisting}[language=MyTex]
$\backslash$begin{lstlisting}
Area($\$$r$\$$) = $\$\backslash$pi r $\string^$ 2$\$$
$\backslash$end{lstlisting}
\end{lstlisting}
Will render like this:
\begin{lstlisting}
Area($r$) = $\pi r^2$
\end{lstlisting}

\subsection{Logic}
The logic preamble contains useful features for writing LTL (Linear Temporal Logic) and CTL (Computation Tree Logic) amongst other things. Use the \T{logic} preamble at the top of your document:
\begin{lstlisting}[language=MyTex]
% LTL, CTL and CTL*
\newcommand{\eventually}{\ensuremath{\Diamond}}
\newcommand{\always}{\ensuremath{\Box}}
\newcommand{\until}{\text{ U }}
\newcommand{\release}{\text{ R }}
\newcommand{\weakuntil}{\text{ W }}
\newcommand{\nex}{\ensuremath{\bigcirc}}
\newcommand{\true}{\text{true}}
\newcommand{\false}{\text{false}}

% Similarity and Bisimilarity
\newcommand{\simulatedby}{\ensuremath{\preceq}}
\newcommand{\similar}{\ensuremath{\simeq}}
\newcommand{\bisimilar}{\ensuremath{\sim}}
 \end{lstlisting}

\block{LTL and CTL}{
  \customfeature{Eventually}{$\backslash$eventually}{$\backslash$eventually}{\eventually}
  \customfeature{Always}{$\backslash$always}{$\backslash$always}{\always}
  \customfeature{Until}{$\backslash$until}{$\backslash$until}{\until}
  \customfeature{Weak Until}{$\backslash$weakuntil}{$\backslash$weakuntil}{\weakuntil}
  \customfeature{Release}{$\backslash$release}{$\backslash$release}{\release}
  \customfeature{Next}{$\backslash$nex}{$\backslash$nex}{\nex}
  \customfeature{True}{$\backslash$true}{$\backslash$true}{\true}
  \customfeature{False}{$\backslash$false}{$\backslash$false}{\false}
}

\block{Similarity and Bisimilarity}{
  \customfeature{Similarity}{$\backslash$similar}{$\backslash$similar}{\similar}
  \customfeature{Bisimilarity}{$\backslash$bisimilar}{$\backslash$bisimilar}{\bisimilar}
  \customfeature{Simulated By}{$\backslash$simulatedby}{$\backslash$simulatedby}{\simulatedby}
}

\subsection{Protocols}
The protocols preamble contains a useful macro for drawing protocols. Use the \T{protocols} preamble at the top of your document:
\begin{lstlisting}[language=MyTex]
% Crytpographic protocols
\usepackage{msc}
\usepackage[normalem]{ulem}

% fix msc package not being able to display math characters
\makeatletter
\def\msc@selfmess#1#2#3{
  \IfStrEq{\mscget{label position}}{left}{\xdef\msc@tempc{-1}}{\xdef\msc@tempc{1}}
  \xdef\msc@tempb{[\mscget{label position},/msc,message,message loop={#3},
            \ifmsc@isstar replay,\fi, inner sep=0pt,
           /tikz/pos=\mscget{pos}, every message, \msc@options]
           (#1)
              to node[xshift=\msc@tempc*\mscget{label distance}](msc@lastnode){\unexpanded\expandafter{\msc@mess@name}}
           (#2);
          }
}

\newcommand{\protocol}[3][]{
  \begin{center}
    \drawframe{no}
    \begin{msc}[msc keyword=, instance distance=15em, environment distance=10em, head top distance=5em, label distance=0.2em]{Protocol: \textbf{#2}\\\ \\ \hfill Public: #1 \hfill \mbox{}}
      #3
    \end{msc}
  \end{center}
}
\newcommand{\party}[2]{\declinst[]{#1}{}{#2}}
\newcommand{\knows}[3]{
  \mess[label position=#3, side=#3]{#2}{#1}{#1}[0]
  \nextlevel
}
\newcommand{\cond}[2]{
  \condition{#1}{#2}
  \nextlevel[2]
}
\newcommand{\msg}[3]{
  \mess{#2}{#1}{#3}
  \nextlevel
}
 \end{lstlisting}

\subsubsection{Protocol Diagrams}
The \T{protocols} preamble contains a macro for drawing protocol diagrams. One can define a protocol diagram like so:
\begin{lstlisting}[language=MyTex]
\protocol[$\$$x$\$$]{ZK from $\$\backslash$Sigma$\$$-protocols}{
  \party{P}{Prover}
  \party{V}{Verifier}

  \knows{V}{$\$$G(k) $\backslash$mapsto (x', w') $\backslash$in R$\$$}{right}
  \msg{V}{$\$$x'$\$$, proof that $\$$V$\$$ knows $\$$w'$\$$}{P}

  \knows{P}{Accept/Reject proof}{left}
  \msg{P}{Proof that $\$$P$ knows $\$$w$\$$ or $\$$w'$\$$}{V}
}\end{lstlisting}
The result is a protocol diagram like this:
\protocol[$x$]{ZK from $\Sigma$-protocols}{
  \party{P}{Prover}
  \party{V}{Verifier}

  \knows{V}{$G(k) \mapsto (x', w') \in R$}{right}
  \msg{V}{$x'$, proof that $V$ knows $w'$}{P}

  \knows{P}{Accept/Reject proof}{left}
  \msg{P}{Proof that $P$ knows $w$ or $w'$}{V}
}
\block{Features}{
  \customfeature{Protocol}{$\backslash$protocol\ \ \ \ \ [public]\ \ \ \ \ \{name\}\ \ \ \ \{body\} \ \ \ \ \ \ \ public is optional.}{$\backslash$protocol[public]\{name\}\ \ \ \ \ \{$\backslash$party\{P\}\{Prover\}\ \ \ \ \ \ \  $\backslash$party\{V\}\{Verifier\}\}}{See following.}
  \customfeature{Party \ \ \ \ \ \ \ \ \ \ \ \ \ \ \ \ \ \ \ (use in body)}{$\backslash$party\ \ \ \ \ \ \ \ \{symbol\}\ \ \{name\}}{$\backslash$party\{P\}\{Prover\}}{See following.}
  \customfeature{Knows  \ \ \ \ \ \ \ \ \ \ \ \ \ \ \ \ \ \ \ (use in body)}{$\backslash$knows\ \ \ \ \ \ \ \ \{symbol\}\ \ \{message\}\ \{side\}}{$\backslash$knows\{V\}\ \ \ \ \ \ \ \ \ \ \ \ \ \ \ \ \ \ \ \{something private\}\ \ \ \ \{left\}}{See following.}
  \customfeature{Message \ \ \ \ \ \ \ \ \ \ \ \ \ \ \ \ (use in body)}{$\backslash$msg\{from\}\ \ \ \ \{message\}\ \{to\}}{$\backslash$msg\{V\}\{hello\}\{P\}}{See following.}
  \customfeature{Condition \ \ \ \ \ \ \ \ \ \ \ \ \ \ \ \ (use in body)}{$\backslash$cond\ \ \ \ \ \ \ \ \ \{message\}\ \{parties\}}{$\backslash$cond\{Wait for P.\}\{P,V\}}{See following.}
}
\B{Protocol example:}
\protocol[public]{name}{
  \party{P}{Prover}
  \party{V}{Verifier}
}
\B{Party example:}
\protocol[public]{name}{
  \party{P}{Prover}
}
\B{Knows example:}
\protocol[public]{name}{
  \party{V}{Verifier}
  \knows{V}{something private}{left}
}
\B{Message example:}
\protocol[public]{name}{
  \party{P}{Prover}
  \party{V}{Verifier}
  \msg{V}{hello}{P}
}
\B{Condition example:}
\protocol[public]{name}{
  \party{P}{Prover}
  \party{V}{Verifier}
  \cond{Wait for P.}{P,V}
}


\pagebreak 
\bibliography{refs}

\end{document}